\documentclass{article}

\usepackage{listings}
\usepackage{hyperref}

\newcommand{\tl}{\textless}
\newcommand{\tg}{\textgreater}

\title{Neovim Configuration}
\author{David Schmid}
\date{May 29, 2023}

\begin{document}
\maketitle

\section{About Neovim}

Neovim is a powerful, extensible, and highly customizable text editor built upon 
the foundations of Vim. This configuration file for Neovim ('init.lua') allows 
to tailor the editor to specific needs and preferences. You can unleash the 
power by customizing key mappings that resonate with your style, channeling 
the raw energy of plugins and sculpting the appearance. With each strum of 
configuration you transcend the ordinary editor to greatness.

By leveraging the Lua programming language, Neovim allows you to command the
editor with unparalleled finesse. This manuscript ignits the flame of your 
Neovim journey, serving as a catalyst of boundless exploration. So go forth, 
fearless adventurer and may your Neovim journey be a symphony of joy and 
discovery.

\section{Installation and Configuration}

\subsection{Installation}

\subsubsection{Installing Neovim}
To install Neovim, visit the official Neovim GitHub page at 

\href{https://github.com/neovim/neovim/wiki/Installing-Neovim}{https://github.com/neovim/neovim/wiki/Installing-Neovim} 

and follow the instructions provided for your specific platform.

For Ubuntu users, it's advised against using the `apt` package manager for 
installing Neovim as the supported version tends to lag behind. Instead, 
consider using the `snap` package manager with the following command:
\begin{verbatim}
sudo snap install nvim --classic
\end{verbatim}
This provides a more up-to-date version of Neovim. Alternatively, for the 
latest version, consider building Neovim from the source code. As of writing, 
the current version in use is 0.9.0.

\subsubsection{Installing Dependencies}
Before proceeding with the configuration, ensure that the required dependencies 
are installed on your system. These are detailed in Section \ref{sec:dependencies}.

\subsubsection{Creating Configuration Directory}
Create a directory in your system's configuration folder where the `init.lua` 
file will be placed:
\begin{verbatim}
mkdir -p ~/.config/nvim
\end{verbatim}

\subsubsection{Creating Symbolic Links}
Create symbolic links from `~/.config/nvim/init.lua` and `~/.config/nvim/lua` 
pointing to the `~/dotfiles/nvim` directory:
\begin{verbatim}
ln -s ~/dotfiles/nvim/init.lua ~/.config/nvim/init.lua
ln -s ~/dotfiles/nvim/lua ~/.config/nvim/lua
\end{verbatim}

\subsubsection{Obtaining Configuration Files}
Get the contents from the GitHub repository at 

\href{https://github.com/5n00py/dotfiles}{https://github.com/5n00py/dotfiles}.

\subsubsection{Launching Neovim}
Start Neovim by running the `nvim` command in the terminal. Neovim will 
automatically load the `init.lua` file and apply the specified configurations.

\subsubsection{Customizing Neovim}
Open the `init.lua` file in Neovim and make any desired modifications to the 
settings, key mappings, and plugin configurations to tailor the editor to your 
preferences.

\subsubsection{Installing Plugins}
After launching Neovim with the custom `init.lua` file, execute the `:Lazy` 
command to trigger the installation and configuration of any lazily loaded 
plugins specified in this `init.lua` file. The command ensures that all 
necessary plugins are installed, and their respective configurations are applied.

\subsubsection{Checking System Health}
After launching Neovim with the custom `init.lua` file, run the `:checkhealth` 
command to analyze the setup and identify potential issues. Review the output, 
and resolve any detected problems according to the provided suggestions or by 
referring to the relevant plugin documentation.

\subsection{Dependencies}\label{sec:dependencies}

This configuration for Neovim requires several external dependencies for 
optimal performance and extended functionality. Ensure that you have the 
following dependencies installed on your system:

\begin{itemize}

\item \textbf{Ripgrep:} A line-oriented search tool that recursively searches 
the current directory for a regex pattern. This is required for plugins like 
Telescope.

\item \textbf{FD:} A simple, fast, and user-friendly alternative to the 'find'
command. This is required for Telescope.

\item \textbf{Build Essentials:} Basic development tools. At least 'make' 
should be installed for Telescope.

\item \textbf{Clipboard Tool (xclip, pbcopy, etc.):} For clipboard support, a 
clipboard tool is required. 'xclip' is commonly used on Linux, while 'pbcopy' 
and 'pbpaste' are used on MacOS. Install via your package manager. For instance, 
'sudo apt install xclip' can be used on Ubuntu.

\item \textbf{Git:} Many Neovim plugins are hosted on GitHub, and Git is 
required to clone and update them.

\item \textbf{Curl, Unzip, Gzip, Tar:} These utilities are often required for 
downloading and extracting files.

\item \textbf{Plenary:} A Lua library used by several Neovim plugins for 
utility functions. It is installed as part of the plugin installation process.

\item \textbf{LuaRocks:} The package manager for Lua modules. Some plugins may 
require Lua modules that can be installed using LuaRocks. Install via package 
manager. For instance, 'sudo apt install luarocks' can be used on Ubuntu.

\item \textbf{Python Modules:} Neovim uses Python3 for some of its plugins and 
features. Specific requirements will be updated as necessary. 

\item \textbf{LSP-related Dependencies:} For the Language Server Protocol (LSP) 
to function, you need to have the respective language server installed for each 
programming language you plan to use in Neovim. These include:
\begin{itemize}
\item Rust: Rust Analyzer
\item Python: Pyright
\item C: Clangd
\item JavaScript and TypeScript: Tsserver
\item Ruby: Solargraph
\item Go: Gopls
\item Shell: Bashls
\end{itemize}
Each language server has its own installation instructions that may require 
some research.

\end{itemize}

\subsection{Keybind Overview}\label{sec:keybind_overview}

This section provides an overview of the most commonly used key mappings in 
this Neovim configuration. This is not an exhaustive list. You can find a 
complete list of mappings by running the ':map' command or by pressing the 
\tl leader\tg\ key and \tl backspace\tg\ to show the Which-Key keybinds popup.

The Which-Key plugin provides a visual representation of key mappings. When 
you press \tl leader\tg\ followed by the \tl backspace\tg\ key, a popup window 
will appear, displaying a hierarchical menu of key mappings and their associated 
actions. 
As you start typing a key combination, Which-Key dynamically generates a popup 
window showing all available options and their associated actions, allowing 
you to navigate through the mappings with ease.

The \tl leader\tg\ key is set to the spacebar by default.

\subsubsection{Tab Management}

\begin{tabular}{|l|l|l|}
\hline
Key & Description & Mode \\
\hline
\tl tab\tg n & Open a new tab with a new empty buffer & Normal \\
\tl tab\tg \tl tab\tg & Go to the next tab & Normal \\
\tl tab\tg p & Go to the previous tab & Normal \\
\tl tab\tg t & Open a new tab with a terminal in insert mode & Normal \\
\tl tab\tg c & Close the current tab & Normal \\
\hline
\end{tabular}

Tab navigation key mappings are associated with the \tl tab\tg\ key. In case the 
\tl tab\tg\ key is already associated with a different action in normal mode, 
you may need to adjust the key mappings to ensure smooth tab navigation within 
Neovim.

\subsubsection{Window Management}

\begin{tabular}{|l|l|l|}
\hline
Key & Description & Mode \\
\hline
\tl leader\tg w & Switch windows & Normal \\
\tl leader\tg q & Switch back to previously used window & Normal \\
\tl leader\tg / & Open empty window on the right & Normal \\
\tl leader\tg - & Open empty window below & Normal \\
\tl leader\tg J & Resize window 5 cells down & Normal \\
\tl leader\tg K & Resize window 5 cells up & Normal \\
\tl leader\tg H & Resize window 5 cells left & Normal \\
\tl leader\tg L & Resize window 5 cells right & Normal \\
\tl leader\tg \tl Left\tg & Resize window 5 cells left & Normal \\
\tl leader\tg \tl Right\tg & Resize window 5 cells right & Normal \\
\tl leader\tg \tl Down\tg & Resize window 5 cells down & Normal \\
\tl leader\tg \tl Up\tg & Resize window 5 cells up & Normal \\
\tl leader\tg h & Select window on the left & Normal \\
\tl leader\tg l & Select window on the right & Normal \\
\tl leader\tg j & Select window below & Normal \\
\tl leader\tg k & Select window above & Normal \\
\tl leader\tg e & Toggle NERDTree window on and off & Normal \\
\tl leader\tg ct & Change Transparency & Normal \\
\hline
\end{tabular}

\subsubsection{Editing}

\begin{tabular}{|l|l|l|}
\hline
Key & Description & Mode \\
\hline
\tl leader\tg a & Select all text in a buffer & Normal \\
\tl leader\tg d & Delete text without yanking & Visual \\
\tl leader\tg p & Paste text without yanking & Visual \\
\tl leader\tg tc & Toggle code comment & Normal / Visual \\
\tl leader\tg ta & Toggle autocompletion (cmp) & Normal / Visual \\
\tl leader\tg n & Edit nvim init.lua & Normal \\
\tl leader\tg u & Toggle Undotree & Normal \\
\hline
\end{tabular}

Neovim provides a wide range of standard editing functions and commands 
to manipulate and modify text. These include operations like copying, cutting, 
pasting, undoing, and redoing. You can access these functions through intuitive 
key combinations and commands, making it efficient to edit and manipulate text 
within Neovim. Familiarize yourself with these standard editing functions to 
enhance your editing experience in Neovim.

\subsubsection{Telescope}

\begin{tabular}{|l|l|l|}
\hline
Key & Description & Mode \\
\hline
\tl leader\tg ff & Find files & Normal \\
\tl leader\tg fg & Live grep & Normal \\
\tl leader\tg fb & List buffers & Normal \\
\tl leader\tg fh & Help tags & Normal \\
\tl leader\tg fw & Grep search on a WORD & Normal \\
\hline
\end{tabular}

\subsubsection{LSP}

\begin{tabular}{|l|l|l|}
\hline
Key & Description & Mode \\
\hline
gd & Go to definition & Normal \\
gD & Go to declaration & Normal \\
K & Show hover information & Normal \\
gr & Find references & Normal \\
\tl leader\tg fmt & Format code & Normal \\
\tl leader\tg d & Show diagnostics under cursor & Normal \\
\hline
\end{tabular}

\subsubsection{Custom Commands}

\begin{tabular}{|l|l|}
\hline
Format & Description \\
\hline
:Rep \tl c\tg \tl w\tg & Repeat \tl c\tg\ times word(s) \tl w\tg\ and insert them to the buffer \\
\hline
\end{tabular}

\section{References and External Resources}\label{sec:references}

This section presents some external resources that can be useful in 
understanding and working with Neovim.

\subsection{External Resources}

\begin{itemize}
\item Neovim Documentation: \href{https://neovim.io/doc/}{https://neovim.io/doc/}
\item Lua Documentation: \href{https://www.lua.org/docs.html}{https://www.lua.org/docs.html}
\item Awesome Neovim: A curated collection of awesome plugins, themes, and resources for Neovim: \href{https://github.com/rockerBOO/awesome-neovim}{https://github.com/rockerBOO/awesome-neovim}
\item Neovim GitHub Repository: \href{https://github.com/neovim/neovim}{https://github.com/neovim/neovim}
\item Neovim subreddit: \href{https://www.reddit.com/r/neovim/}{https://www.reddit.com/r/neovim/}
\end{itemize}

\subsection{Manual Pages}

Neovim provides a built-in help system which can be accessed using the ':help' 
command. Below are a few examples of how to use this system:

\begin{itemize}
\item \verb|:help init.lua|: Provides an introduction to customizing.
\item \verb|:h key-notation|: Displays information about key notation.
\item \verb|:h :w|: Views the help page for the write command.
\end{itemize}


\end{document}
