\documentclass{article}

\title{Zsh Configuration}
\author{David Schmid}
\date{May 25, 2023}

\begin{document}
\maketitle

\section{About ZSH}
Zsh is a powerful and feature-rich shell designed for interactive use. It 
extends the Bourne shell syntax and incorporates many useful features from 
shells like bash, ksh, and tcsh.

Compared to other shells, such as Bash, Zsh offers an enhanced interactive 
experience with a wide range of capabilities. It provides advanced tab 
completion, spelling correction, syntax highlighting, and a robust plugin 
system, among other features.

Zsh offers extensive customization options, allowing you to tailor your shell 
environment to suit your needs and preferences. It supports custom aliases, 
functions, keybindings, and prompt configurations, making it highly adaptable 
to individual workflows.

In addition, Zsh has a vibrant and active community that contributes to its 
ecosystem. Popular frameworks like Oh-My-Zsh provide a wealth of preconfigured 
themes, plugins, and helpers to further enhance your Zsh experience.

By leveraging the power and flexibility of Zsh, you can optimize your shell 
workflow and streamline your command-line interactions.

\section{Installation and Configuration}
To begin using Zsh, you'll need to install it and configure it as your 
default shell. Follow these steps:

\subsection{Installation}
Zsh can be installed using a package manager. For example, on a Debian-based 
system, you can use the following command:
\begin{verbatim}
sudo apt install zsh
\end{verbatim}

\subsection{Change Default Shell to Zsh}
After installing Zsh, you need to change your default shell to Zsh. To do this, 
use the following command:
\begin{verbatim}
chsh -s $(which zsh)
\end{verbatim}

Once the command executes successfully, your default shell will be set to Zsh.
Please note that in some cases, you might have to reboot your system for the 
changes to take effect. Rebooting ensures that Zsh is properly initialized as 
your default shell across all sessions and environments.

\subsection{Configuration}
To configure Zsh with your desired settings, you can follow these steps:

\begin{enumerate}
  \item Obtain the zsh configuration files: You can either copy the 
	  'dotfiles/zsh' directory to your home directory or clone the 
		configuration repository.

  \item Note: If your 'dotfiles' directory is not located in the home 
	  directory, you need to adapt the zsh directory path accordingly in the
		'zshrc' file. Update the path in the following line to match the actual 
		location of your 'zsh' directory:
		\begin{verbatim}
		export zsh_dir="$HOME/dotfiles/zsh"
		\end{verbatim}

  \item Backup your existing .zshrc file: Make a backup of your current
	  '.zshrc' file by running the following command:
  		\begin{verbatim}
  		cp ~/.zshrc ~/.zshrc.bak
  		\end{verbatim}

  \item Remove the existing .zshrc file: Run the following command to remove 
	  the existing `.zshrc` file:
  		\begin{verbatim}
  		rm ~/.zshrc
  		\end{verbatim}

  \item Create a symbolic link to the zshrc file: Create a symbolic link that 
	  points to the 'zshrc' file within the 'dotfiles/zsh' directory. For 
		example, run the following command:
  		\begin{verbatim}
  		ln -s ~/dotfiles/zsh/zshrc ~/.zshrc
  		\end{verbatim}

  \item Create a private configuration file: Create a separate file named
	  '~/.zsh-private.zsh' outside the 'dotfiles' directory to store any 
		private configuration settings or sensitive information. Be sure to 
		add this file to your '.gitignore' or any other version control 
		exclusions if needed to prevent accidental exposure.

  \item Restart the terminal: Restart your terminal to load the new Zsh 
	  configuration. Once this is done, the new configuration settings will 
		take effect. After this is done you can reload the configuration at 
		any time by running the command 'zshreload'.
\end{enumerate}

\section{Dependencies}
This zsh configuration relies on several external tools and packages to provide 
additional functionality. Ensure that the following dependencies are installed 
on your system:

\subsection{Oh-My-Zsh}
Oh-My-Zsh is an open-source, community-driven framework for managing the Zsh 
configuration. It provides a vast collection of themes, plugins, and helpers 
that extend the capabilities of Zsh. Install Oh-My-Zsh by following the 
instructions available on the official GitHub repository.

\subsection{Powerlevel9k}
Powerlevel9k is a popular theme for Oh-My-Zsh, offering a highly customizable 
and informative prompt for your shell. Install the Powerlevel9k theme by 
following the installation instructions provided in the official GitHub repository.

\subsection{Tmux}
Tmux is a terminal multiplexer that allows you to manage multiple terminal 
sessions within a single window. It enhances your productivity by providing 
features like session management, window splitting, and customizable keybindings. 
Install Tmux using your package manager of choice.

\subsection{Other Dependencies}

\begin{itemize}
  \item Git: A distributed version control system
  \item Bluetoothctl: A command-line utility for pairing with Bluetooth devices.
  \item Nvim: A highly configurable text editor.
  \item Colordiff: A wrapper for diff with pretty syntax highlighting.
  \item Black: A Python code formatter.
  \item Python3: The Python programming language (version 3.x).
  \item Clang-format: A code formatter for C and other programming languages.
  \item Youtube-dl: A command-line tool for downloading videos from YouTube.
  \item Zathura: A PDF viewer.
  \item Groff: TBD (To Be Determined) - Add any other dependencies specific to your setup.
\end{itemize}

Ensure that these dependencies are installed on your system or remove depending
commands.

\end{document}
