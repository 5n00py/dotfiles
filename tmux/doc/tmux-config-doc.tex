\documentclass{article}

\title{Tmux Configuration}
\author{David Schmid}
\date{May 17, 2023}

\begin{document}

\maketitle

\section{About}
Tmux is a terminal multiplexer, which allows multiple terminal sessions to be 
created, accessed, and managed from a single screen. Tmux may be detached from 
a screen and continue running in the background, then later reattached.

\section{Installation}
To install tmux on Debian-based systems, you can use the package manager:

\begin{verbatim}
sudo apt update
sudo apt install tmux
\end{verbatim}

\section{Configuration}
To use this configuration, follow these steps:

\begin{enumerate}
  \item Clone the dotfiles repository to your local machine:
    \begin{verbatim}
    git clone https://github.com/5n00py/dotfiles.git    
	\end{verbatim}
  \item Create a symbolic link from \texttt{\textasciitilde/.tmux.conf} to \texttt{~/dotfiles/tmux/tmux.conf}:
    \begin{verbatim}
    ln -s ~/dotfiles/tmux/tmux.conf ~/.tmux.conf
    \end{verbatim}
  \item Start a new tmux session.
  \item You may need to install the plugins specified in this configuration file. 
  Use the Tmux Plugin Manager (TPM) to install the plugins by 
  pressing \texttt{prefix + I} within tmux.
\end{enumerate}

\section{Dependencies}
The following dependencies are required to use this tmux configuration:
\begin{itemize}
  \item tmux: Terminal multiplexer.
  \item Tmux Plugin Manager (TPM): Plugin manager for tmux.
  \item xclip: Command-line interface to the X11 clipboard.
  \item tree: Recursive directory listing program.
  \item tmux-sensible: Set of sensible tmux options.
  \item tmux-resurrect: Save and restore tmux sessions.
  \item tmux-continuum: Continuous saving of tmux environment.
  \item tmux-net-speed: Display network speed in the status bar.
  \item tmux-cpu: Display CPU usage and load in the status bar.
  \item tmux-sidebar: Quick navigation through files and directories.
\end{itemize}

\section{Keybind Overview}
\subsection{Prefix}
\begin{itemize}
  \item \texttt{<Ctrl-Space>} - Main prefix key for all tmux commands
\end{itemize}

\subsection{General}
\begin{itemize}
  \item \texttt{prefix + r} - Reload configuration file
  \item \texttt{prefix + s} - Toggle the status bar on and off
  \item \texttt{prefix + q} - Detach from the current tmux session
  \item \texttt{prefix + Tab} - List active sessions to choose one to switch to
\end{itemize}

\subsection{Pane Management}
\begin{itemize}
  \item \texttt{prefix + /} - Split window horizontally
  \item \texttt{prefix + |} - Split window horizontally
  \item \texttt{prefix + 7} - Split window horizontally
  \item \texttt{prefix + -} - Split window vertically
  \item \texttt{M-Left} - Select pane to the left
  \item \texttt{M-Right} - Select pane to the right
  \item \texttt{M-Up} - Select pane above
  \item \texttt{M-Down} - Select pane below
  \item \texttt{prefix + h} - Select pane to the left
  \item \texttt{prefix + l} - Select pane to the right
  \item \texttt{prefix + j} - Select pane below
  \item \texttt{prefix + k} - Select pane above
  \item \texttt{prefix + J} - Resize the current pane 5 cells down (non-repeatable)
  \item \texttt{prefix + K} - Resize the current pane 5 cells up (non-repeatable)
  \item \texttt{prefix + H} - Resize the current pane 5 cells to the left (non-repeatable)
  \item \texttt{prefix + L} - Resize the current pane 5 cells to the right (non-repeatable)
  \item \texttt{prefix + Left Arrow} - Resize the current pane 5 cells to the left (repeatable)
  \item \texttt{prefix + Right Arrow} - Resize the current pane 5 cells to the right (repeatable)
  \item \texttt{prefix + Up Arrow} - Resize the current pane 5 cells up (repeatable)
  \item \texttt{prefix + Down Arrow} - Resize the current pane 5 cells down (repeatable)
\end{itemize}

\subsection{Copy Mode}
\begin{itemize}
  \item \texttt{prefix + v} - Enter copy mode
  \item \texttt{prefix + p} - Paste buffer
  \item \texttt{space} (in copy mode) - Start selection
  \item \texttt{y} (in copy mode) - Copy selection to system clipboard
  \item \texttt{Escape} - Exit copy mode (vi mode)
\end{itemize}

\subsection{Plugins}
\begin{itemize}
  \item \texttt{prefix + I} - Install new plugins (TPM command)
  \item \texttt{prefix + w} - Save the current tmux session (tmux-resurrect)
  \item \texttt{prefix + Tab} - Toggle sidebar with a directory tree (tmux-sidebar)
  \item \texttt{prefix + Backspace} - Toggle sidebar and move cursor to it (tmux-sidebar)
\end{itemize}

\section{References}
\subsection{External Resources}
\begin{itemize}
  \item tmux Guide: \texttt{https://tmuxguide.readthedocs.io}
  \item tmux Github: \texttt{https://github.com/tmux/tmux}
  \item tmuxcheatsheet: \texttt{https://tmuxcheatsheet.com}
  \item tmux Plugin Manager (TPM): \texttt{https://github.com/tmux-plugins/tpm}
\end{itemize}

\subsection{Manual Pages}
For a comprehensive list of all available tmux commands and their 
descriptions, refer to the man pages by typing \texttt{man tmux} into 
your terminal.

\subsection{Useful Commands}
\begin{itemize}
  \item Start new session: \texttt{tmux new -s [session name]}
  \item Attach to existing session: \texttt{tmux attach -t [session name]}
  \item List all sessions: \texttt{tmux ls}
  \item Detach from current session: \texttt{tmux detach}
  \item Kill session: \texttt{tmux kill-session -t [session name]}
\end{itemize}

For a full list of commands and key bindings in tmux, you can use the 
command-prompt within a tmux session: \texttt{prefix + :list-commands} 
and \texttt{prefix + :list-keys}.

\end{document}
